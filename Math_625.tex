\documentclass[11pt]{article}
\usepackage[left=.7in,right=.7in,top=.7in,bottom=.7in]{geometry}
%Set up the margins
\usepackage{amssymb,amsmath,amsthm,mathrsfs,verbatim}
\usepackage{extarrows}%% For extending the equal sign when there is   \xlongequal{<stuff>}
%\sout{} strike through!
\usepackage{marginnote}%\marginnote{}
\usepackage[colorlinks,linkcolor=blue]{hyperref}

%Create a header at the top of every page
\usepackage{fancyhdr}
\pagestyle{fancy}

\usepackage{pdfpages} %merge pdf with: \includepdfmerge{heine-borel_proof.pdf,-}
\usepackage{graphicx}%to input images \includegraphics[width=130mm]{Open_Open}
\usepackage{lipsum}
\usepackage{color}

%You can define commands for the things you use frequently.
\newcommand{\al}{\aleph}
\newcommand{\A}{\mathcal{A}}
\newcommand{\D}{\mathcal D}
\renewcommand{\c}{\mathfrak c}
\newcommand{\C}{{\mathbb C}}
\renewcommand{\L}{{\mathcal L}}
\newcommand{\M}{\mathcal M}
\newcommand{\N}{{\mathbb N}}
\renewcommand{\P}{\mathcal {P}}
\newcommand{\Q}{{\mathbb Q}}
\newcommand{\R}{{\mathbb R}}
\newcommand{\X}{{\mathlarger{\mathcal X}}}
\newcommand{\Z}{{\mathbb Z}}
\newcommand{\To}{\Rightarrow}
\newcommand{\WLOG}{With out loss of generality}
\newcommand{\card}{\text{card }}

% The limits.
\newcommand{\limk}{\underset{k\to\infty}\lim}
\newcommand{\limn}{\underset{n\to\infty}\lim}
\newcommand{\liminfn}{\underset{n\to\infty}{\underline{\lim}}}
\newcommand{\liminfk}{\underset{k\to\infty}{\underline{\lim}}}
\newcommand{\liminfp}{\underset{p\to\infty}{\underline{\lim}}}
\newcommand{\liminfj}{\underset{j\to\infty}{\underline{\lim}}}
\newcommand{\limsupn}{\underset{n\to\infty}{\overline{\lim}}}
\newcommand{\limsupk}{\underset{k\to\infty}{\overline{\lim}}}
\newcommand{\limsupp}{\underset{p\to\infty}{\overline{\lim}}}
\newcommand{\limsupr}{\underset{r\to\infty}{\overline{\lim}}}


\newcommand{\sgn}{\text{sgn}}

% For L^p space
\newcommand{\Linfty}{{L^\infty(E)}}
\newcommand{\Lp}{{L^{p}(E)}}
\newcommand{\Lq}{{L^{q}(E)}}

% 偷懒
\renewcommand{\phi}{\varphi}
\renewcommand{\epsilon}{\varepsilon}


\usepackage{mdframed}%needed for box like theorems.\newmdtheoremenv %In the box, the footnotes are more handy! ^.^

\newtheorem{theorem}{Theorem}[subsection]
\newtheorem{acknowledgement}[theorem]{Acknowledgement}
\newtheorem{algorithm}[theorem]{Algorithm}
\newtheorem{case}[theorem]{Case}
\newtheorem{claim}[theorem]{Claim}
\newmdtheoremenv{claimbox}[theorem]{Claim}
\newtheorem{conclusion}[theorem]{Conclusion}
\newtheorem{condition}[theorem]{Condition}
\newmdtheoremenv{conjecture}[theorem]{Conjecture}
\newtheorem{corollary}[theorem]{Corollary}
\newtheorem{criterion}[theorem]{Criterion}
\newmdtheoremenv{definition}[theorem]{Definition}
\newtheorem*{defn}{Definition}
\newtheorem{example}[theorem]{Example}
\newtheorem{exercise}[theorem]{Exercise}
\newtheorem{fact}[theorem]{fact}
\newmdtheoremenv{idea}[theorem]{Idea}
\newtheorem{lemma}[theorem]{Lemma}
\newtheorem{notation}[theorem]{Notation}
\newtheorem{question}[theorem]{Question}
\newmdtheoremenv{question_sqrt}{Question}
\newtheorem{proposition}[theorem]{Proposition}
\newtheorem*{problem}{\textcolor[rgb]{1.00,0.00,0.00}{Problem}}
\newtheorem{remark}[theorem]{Remark}
\newmdtheoremenv{remark_box}[theorem]{Remark}
\newtheorem{solution}[theorem]{Solution}
\newtheorem{summary}[theorem]{Summary}
\newmdtheoremenv{setup}{Set Up}
\newmdtheoremenv{joke}{Joke}
\newmdtheoremenv{typo}{Typo Correction}
\newtheorem*{rudin}{Rudin Says}
%Textbook: elementary  : the theory of calculus
\title{Math 625\\
Professor Erhan Bayraktar}
\author{Linfeng Li \\ llinfeng@umich.edu}

\usepackage{mathrsfs} % enable people to use \mathscr{A}
\usepackage{color}
\usepackage{xcolor}
\newcommand{\hilight}[1]{\colorbox{yellow}{#1}}
\usepackage{bbm}
\usepackage{relsize} % For large symbles: \mathlarger{math_expression}
\usepackage{marvosym} % 笑脸
%\usepackage{enumitem} % For using: \begin{itemize}[leftmargin=-.5in]
\usepackage{enumerate}
%\usepackage{comment}

% To make footnote numering by section.
\makeatletter
\@addtoreset{footnote}{section}
\makeatother
\usepackage{colonequals}
\begin{document}
\maketitle
\section{Week 1}
\subsection{Tue: 2014-09-02}
\subsubsection{Basics}
Exam (in class, 90 minutes): Tue, Oct 7; Tue, Nov 11; Tue, Dec 9.   || 25\%, 30
\% and 40 \%.  
5\% attendance. 


\textbf{Textbook: } Probability and Statistics (took the class from the writer,
available through library).  
Probability with \textbf{Martingale}, the latter is the emphasis.

\textbf{Exercises:} Book has exercises, but not graded homework.

\subsubsection{Beginning: measure theory}
\paragraph{Sigma-algebra}
Given a set $E$ (a universal set), $\mathcal E$, a nonempty collection of
subsets of $E$, is called a $\sigma$-algebra if closed under complements \&
countable unions. 

\begin{itemize}
  \item The most trivial sigma-algebra  $\{\emptyset, E\}$ is called the trivial
$\sigma$-algebra. 
\end{itemize}

\begin{definition}[$\sigma$ algebra generated by $\mathcal C$]
  Given a collection $\mathcal C$ of subsets (of $E$), $\sigma(\mathcal C)$ will
  denote the smaller $\sigma$-algebra containing $\mathcal C$. 
\end{definition}

\begin{definition}
  $\sigma$-algebra generated by open sets  is called \textbf{Borel
  $\sigma$-algebra}.
\end{definition}

\paragraph{$p$-system} A collection of $\mathcal C$ which is closed under
(finite) intersection.

``p'' for product, could also use $\pi$-system. The latter in Greek. 
\paragraph{$d$-system} A collection $\mathcal D$ is called a $d$-system if 
\begin{enumerate}[(i)]
  \item $E\in \mathcal D$, 
  \item $A, B \in \mathcal D$, $A\supset B$ $\implies$ $A\setminus B \in \mathcal D$
  \item $\left( A_n \right) \subset \mathcal D$ and $A_n \uparrow A$ $\implies$
    $A\subset \mathcal D$. 
\end{enumerate}
(``d'' for Dynkin)




Note: curly characters are for collection of sets. 
\begin{proposition}
  $\mathcal E$ is a $\sigma$-algebra if and only if it is a $p$-system and a
  $d$-system. 
% A sigma algebra is both a $p$-system and a $d$ system.
  \begin{proof}
    $\Rightarrow$ is trivial.

    $\Leftarrow$: Let $\mathcal E$ be a collection which is a $p$-system and a
    $d$-system. 
    \begin{enumerate}
      \item Closed under complements (to be a sigma algebra).
        Let $A\in  \mathcal E$ $\implies$ $E\setminus A \in \mathcal E$ by
        $(ii)$ for property of $d$-system. 
      \item Closed under finite unions: $A, B \in \mathcal E$ $\implies$
        $A\bigcup B = \left( A^c \cap B^c \right)^c$ by $1$ above and property
        of $p$-system.
      \item Closed under countable unions: for $\left( A_n \right) \subset
        \mathcal E$, $\bigcup _n A_n \in \mathcal E$? We construct an
        increasing sequence of $\left( B_n \right)$: 

        Let $B_1 = A_1$, $B_2 = A_1 \bigcup A_2 \in \mathcal E$ \dots
        $\bigcup _ n A_n = \bigcup _ n B_n $. Then by $(iii)$ for property of
        $d$-system, the conclusion follows.
    \end{enumerate}
  \end{proof}
  
\end{proposition}

\begin{lemma}
  For $\mathcal D$, a $d$-system, fix $D\in \mathcal D$. Define $\hat { \mathcal D}
  := \left\{ A \in \mathcal D : A\cap D \in \mathcal D \right\}$. Then, $\hat
  {\mathcal D} $ is also a $d$-system.
  \label{hw:homework-week1}
\end{lemma}

\paragraph{Monotone Class Theorem}
[Very useful tool in showing an arbitrary collection of sets is a $\sigma$-algebra]

\begin{theorem}
  If a $d$-system contains a $p$-system, then it also contains the
  $\sigma$-algebra generated by the $p$-system.
  \label{thm:monotone-class-theorem}
  \begin{proof}
    Symbolic expression: $\mathcal C \subset \mathcal D \implies \sigma
    (\mathcal C) \subset \mathcal D$.

    \textbf{Step 1: }


    Let $\mathcal C$ be a $p$-system. $\mathcal D$ is the smallest $d$-system
    that contains $\mathcal C$. 
    \footnote{(***: little result -- the intersections of
    $d$-systems is a $d$-system [to obtain the ``smallest'']. Also, the
  ``smallest'' matters.)}

    Enough to show $\mathcal D \supset \sigma (\mathcal C)$. 

    If fact, we will show $\mathcal D$ is a $\sigma$-algebra. By proposition, it
    is enough to show it is a $p$-system. 

    \textbf{Step 2: }

    Fix $B \in \mathcal C$ and let $\mathcal D_1 := \left\{ A \in \mathcal D : A
    \bigcap B \in \mathcal D\right\}$. 1) By the lemma \ref{hw:homework-week1},
    $\mathcal D_1$ is a $d$-system. 2) $\mathcal C \subset \mathcal D_1$. 

    1) and 2) $\implies $ $\mathcal D_1 = \mathcal D$.


    \textbf{Setp 3: }

    Fix $A\in \mathcal D$, let $\mathcal D_2 := \left\{ B \in \mathcal D :
    B\bigcap A \in \mathcal D \right\}$. 

    1) by the lemma, $\mathcal D_2$ is a $d$-system. 2) by \textbf{step 2},
    $\mathcal C \subset D_2$. 

    1) and 2) $\implies$ $\mathcal D_2 = \mathcal D$. 



    \textbf{Step 1-3} gives that $\mathcal D$ is a $p$-system.




    In here, $\mathcal D = \sigma ( \mathcal C)$. [But in the theorem, this is
    not a necessary conclusion.]
  \end{proof}
\end{theorem}

\paragraph{Measurable space }
$(E, \mathcal E)$ is a measurable space. [$\mathcal E$ is a $\sigma$-algebra on
$E$.]

\paragraph{Products of measure spaces} $(E, \mathcal E)$, $(F, \mathcal F)$.
Then $\left( E \times F, \mathcal E light-product \mathcal F \right)$ where
$\times$ is regular set product; and the light-product is $\sigma \left( \text{
generated by measurable rectangles} 
\right)$

\paragraph{Measurable functions (random variables)}

\begin{lemma}
  A mapping $f: E\to F$ and (inverse mapping) $f^{-1} (A) := \left\{ x \in E :
    f(x) \in A
  \right\}.$ Then, $f^{-1} \emptyset = \emptyset$. $f^{-1}(F) = E$. $f^{-1}
  (B\setminus C) = f^{-1} (B) \setminus f^{-1} (C)$. 
  $f^{-1}\left( \bigcup _ i B_i \right) = \bigcup _i f^{-1}(B_i)$ and $f^{-1}\left( \bigcap B_i \right) =
  \bigcap _ i f^{-1}\left( B_i \right)$.


  (set operation passes through the inverse function operation.)
  \label{lemma:measurable-function}
\end{lemma}


\begin{definition}
  $(E, \mathcal E)$, $(F, \mathcal F)$. $f: E\to F$ is "measurable" relative to
  $\mathcal E \& \mathcal F$ if $f^{-1}(B) \in \mathcal E$, $\forall B \in
  \mathcal F$. 
\end{definition}




\end{document} 
\substack 
cmiless  <-- the Physics PhD student at Umich.

\hrule
\smallskip
\hruleSince

$\ddot\smile$ %Smiling Face
$ \lfloor \rfloor$ %Tilman's special braces.


\hrule
\smallskip
\hruleSince

$\ddot\smile$ %Smiling Face
$ \lfloor \rfloor$ %Tilman's special braces.
