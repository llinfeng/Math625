\documentclass[11pt]{article}
\usepackage[left=.7in,right=.7in,top=.7in,bottom=.7in]{geometry}
%Set up the margins
\usepackage{amssymb,amsmath,amsthm,mathrsfs,verbatim}
\usepackage{extarrows}%% For extending the equal sign when there is   \xlongequal{<stuff>}
%\sout{} strike through!
\usepackage{marginnote}%\marginnote{}
\usepackage[colorlinks,linkcolor=blue]{hyperref}

%Create a header at the top of every page
\usepackage{fancyhdr}
\pagestyle{fancy}

\usepackage{pdfpages} %merge pdf with: \includepdfmerge{heine-borel_proof.pdf,-}
\usepackage{graphicx}%to input images \includegraphics[width=130mm]{Open_Open}
\usepackage{lipsum}
\usepackage{color}

%You can define commands for the things you use frequently.
\newcommand{\al}{\aleph}
\newcommand{\A}{\mathcal{A}}
\newcommand{\D}{\mathcal D}
\renewcommand{\c}{\mathfrak c}
\newcommand{\C}{{\mathbb C}}
\renewcommand{\L}{{\mathcal L}}
\newcommand{\M}{\mathcal M}
\newcommand{\m}{\mathcal}
\newcommand{\N}{{\mathbb N}}
\renewcommand{\P}{\mathcal {P}}
\newcommand{\Q}{{\mathbb Q}}
\newcommand{\R}{{\mathbb R}}
\newcommand{\X}{{\mathlarger{\mathcal X}}}
\newcommand{\Z}{{\mathbb Z}}
\newcommand{\To}{\Rightarrow}
\newcommand{\WLOG}{With out loss of generality}
\newcommand{\card}{\text{card }}

% The limits.
\newcommand{\limk}{\underset{k\to\infty}\lim}
\newcommand{\limn}{\underset{n\to\infty}\lim}
\newcommand{\liminfn}{\underset{n\to\infty}{\underline{\lim}}}
\newcommand{\liminfk}{\underset{k\to\infty}{\underline{\lim}}}
\newcommand{\liminfp}{\underset{p\to\infty}{\underline{\lim}}}
\newcommand{\liminfj}{\underset{j\to\infty}{\underline{\lim}}}
\newcommand{\limsupn}{\underset{n\to\infty}{\overline{\lim}}}
\newcommand{\limsupk}{\underset{k\to\infty}{\overline{\lim}}}
\newcommand{\limsupp}{\underset{p\to\infty}{\overline{\lim}}}
\newcommand{\limsupr}{\underset{r\to\infty}{\overline{\lim}}}


\newcommand{\sgn}{\text{sgn}}

% For L^p space
\newcommand{\Linfty}{{L^\infty(E)}}
\newcommand{\Lp}{{L^{p}(E)}}
\newcommand{\Lq}{{L^{q}(E)}}

% 偷懒
\renewcommand{\phi}{\varphi}
\renewcommand{\epsilon}{\varepsilon}


\usepackage{mdframed}%needed for box like theorems.\newmdtheoremenv %In the box, the footnotes are more handy! ^.^

\newtheorem{theorem}{Theorem}[subsection]
\newtheorem{acknowledgement}[theorem]{Acknowledgement}
\newtheorem{algorithm}[theorem]{Algorithm}
\newtheorem{case}[theorem]{Case}
\newtheorem{claim}[theorem]{Claim}
\newmdtheoremenv{claimbox}[theorem]{Claim}
\newtheorem{conclusion}[theorem]{Conclusion}
\newtheorem{condition}[theorem]{Condition}
\newmdtheoremenv{conjecture}[theorem]{Conjecture}
\newtheorem{corollary}[theorem]{Corollary}
\newtheorem{criterion}[theorem]{Criterion}
\newmdtheoremenv{definition}[theorem]{Definition}
\newtheorem*{defn}{Definition}
\newtheorem{example}[theorem]{Example}
\newtheorem{exercise}[theorem]{Exercise}
\newtheorem{fact}[theorem]{fact}
\newmdtheoremenv{idea}[theorem]{Idea}
\newtheorem{lemma}[theorem]{Lemma}
\newtheorem{notation}[theorem]{Notation}
\newtheorem{question}[theorem]{Question}
\newmdtheoremenv{question_sqrt}{Question}
\newtheorem{proposition}[theorem]{Proposition}
\newtheorem*{problem}{\textcolor[rgb]{1.00,0.00,0.00}{Problem}}
\newtheorem{remark}[theorem]{Remark}
\newmdtheoremenv{remark_box}[theorem]{Remark}
\newtheorem{solution}[theorem]{Solution}
\newtheorem{summary}[theorem]{Summary}
\newmdtheoremenv{setup}{Set Up}
\newmdtheoremenv{joke}{Joke}
\newmdtheoremenv{typo}{Typo Correction}
\newtheorem*{rudin}{Rudin Says}
%Textbook: elementary  : the theory of calculus
\title{Math 625\\
Professor Erhan Bayraktar}
\author{Linfeng Li \\ llinfeng@umich.edu}

\usepackage{mathrsfs} % enable people to use \mathscr{A}
\usepackage{color}
\usepackage{xcolor}
\newcommand{\hilight}[1]{\colorbox{yellow}{#1}}
\usepackage{bbm}
\usepackage{relsize} % For large symbles: \mathlarger{math_expression}
\usepackage{marvosym} % 笑脸
%\usepackage{enumitem} % For using: \begin{itemize}[leftmargin=-.5in]
\usepackage{enumerate}
%\usepackage{comment}

% To make footnote numering by section.
\makeatletter
\@addtoreset{footnote}{section}
\makeatother
\usepackage{colonequals}
\begin{document}
\maketitle
No responding to emails!
\section{Week 1}
\subsection{Tue: 2014-09-02}
\subsubsection{Basics}
Exam (in class, 90 minutes): Tue, Oct 7; Tue, Nov 11; Tue, Dec 9.   || 25\%, 30
\% and 40 \%.  
5\% attendance. 


\textbf{Textbook: } Probability and Statistics (took the class from the writer,
available through library).  
Probability with \textbf{Martingale}, the latter is the emphasis.

\textbf{Exercises:} Book has exercises, but not graded homework.

\subsubsection{Beginning: measure theory}
\paragraph{Sigma-algebra}
Given a set $E$ (a universal set), $\mathcal E$, a nonempty collection of
subsets of $E$, is called a $\sigma$-algebra if closed under complements \&
countable unions. 

\begin{itemize}
  \item The most trivial sigma-algebra  $\{\emptyset, E\}$ is called the trivial
$\sigma$-algebra. 
\end{itemize}

\begin{definition}[$\sigma$ algebra generated by $\mathcal C$]
  Given a collection $\mathcal C$ of subsets (of $E$), $\sigma(\mathcal C)$ will
  denote the smaller $\sigma$-algebra containing $\mathcal C$. 
\end{definition}

\begin{definition}
  $\sigma$-algebra generated by open sets  is called \textbf{Borel
  $\sigma$-algebra}.
\end{definition}

\paragraph{$p$-system} A collection of $\mathcal C$ which is closed under
(finite) intersection.

``p'' for product, could also use $\pi$-system. The latter in Greek. 
\paragraph{$d$-system} A collection $\mathcal D$ is called a $d$-system if 
\begin{enumerate}[(i)]
  \item $E\in \mathcal D$, 
  \item $A, B \in \mathcal D$, $A\supset B$ $\implies$ $A\setminus B \in \mathcal D$
  \item $\left( A_n \right) \subset \mathcal D$ and $A_n \uparrow A$ $\implies$
    $A\subset \mathcal D$. 
\end{enumerate}
(``d'' for Dynkin)




Note: curly characters are for collection of sets. 
\begin{proposition}
  $\mathcal E$ is a $\sigma$-algebra if and only if it is a $p$-system and a
  $d$-system. 
% A sigma algebra is both a $p$-system and a $d$ system.
  \begin{proof}
    $\Rightarrow$ is trivial.

    $\Leftarrow$: Let $\mathcal E$ be a collection which is a $p$-system and a
    $d$-system. 
    \begin{enumerate}
      \item Closed under complements (to be a sigma algebra).
        Let $A\in  \mathcal E$ $\implies$ $E\setminus A \in \mathcal E$ by
        $(ii)$ for property of $d$-system. 
      \item Closed under finite unions: $A, B \in \mathcal E$ $\implies$
        $A\bigcup B = \left( A^c \cap B^c \right)^c$ by $1$ above and property
        of $p$-system.
      \item Closed under countable unions: for $\left( A_n \right) \subset
        \mathcal E$, $\bigcup _n A_n \in \mathcal E$? We construct an
        increasing sequence of $\left( B_n \right)$: 

        Let $B_1 = A_1$, $B_2 = A_1 \bigcup A_2 \in \mathcal E$ \dots
        $\bigcup _ n A_n = \bigcup _ n B_n $. Then by $(iii)$ for property of
        $d$-system, the conclusion follows.
    \end{enumerate}
  \end{proof}
  
\end{proposition}

\begin{lemma}
  For $\mathcal D$, a $d$-system, fix $D\in \mathcal D$. Define $\hat { \mathcal D}
  := \left\{ A \in \mathcal D : A\cap D \in \mathcal D \right\}$. Then, $\hat
  {\mathcal D} $ is also a $d$-system.
  \label{hw:homework-week1}
\end{lemma}

\paragraph{Monotone Class Theorem}

[Very useful tool in showing an arbitrary collection of sets is a $\sigma$-algebra]

\begin{theorem}
  If a $d$-system contains a $p$-system, then it also contains the
  $\sigma$-algebra generated by the $p$-system.
  \label{thm:monotone-class-theorem}
  \begin{proof}
    Symbolic expression: $\mathcal C \subset \mathcal D \implies \sigma
    (\mathcal C) \subset \mathcal D$.

    \textbf{Step 1: }


    Let $\mathcal C$ be a $p$-system. $\mathcal D$ is the smallest $d$-system
    that contains $\mathcal C$. 
    \footnote{(***: little result -- the intersections of
    $d$-systems is a $d$-system [to obtain the ``smallest'']. Also, the
  ``smallest'' matters.)}

    Enough to show $\mathcal D \supset \sigma (\mathcal C)$. 

    If fact, we will show $\mathcal D$ is a $\sigma$-algebra. By proposition, it
    is enough to show it is a $p$-system. 

    \textbf{Step 2: }

    Fix $B \in \mathcal C$ and let $\mathcal D_1 := \left\{ A \in \mathcal D : A
    \bigcap B \in \mathcal D\right\}$. 1) By the lemma \ref{hw:homework-week1},
    $\mathcal D_1$ is a $d$-system. 2) $\mathcal C \subset \mathcal D_1$. 

    1) and 2) $\implies $ $\mathcal D_1 = \mathcal D$.


    \textbf{Setp 3: }

    Fix $A\in \mathcal D$, let $\mathcal D_2 := \left\{ B \in \mathcal D :
    B\bigcap A \in \mathcal D \right\}$. 

    1) by the lemma, $\mathcal D_2$ is a $d$-system. 2) by \textbf{step 2},
    $\mathcal C \subset D_2$. 

    1) and 2) $\implies$ $\mathcal D_2 = \mathcal D$. 



    \textbf{Step 1-3} gives that $\mathcal D$ is a $p$-system.




    In here, $\mathcal D = \sigma ( \mathcal C)$. [But in the theorem, this is
    not a necessary conclusion.]
  \end{proof}
\end{theorem}

\paragraph{Measurable space }
$(E, \mathcal E)$ is a measurable space. [$\mathcal E$ is a $\sigma$-algebra on
$E$.]

\paragraph{Products of measure spaces} $(E, \mathcal E)$, $(F, \mathcal F)$.
Then $\left( E \times F, \mathcal E light-product \mathcal F \right)$ where
$\times$ is regular set product; and the light-product is $\sigma \left( \text{
generated by measurable rectangles} 
\right)$

\paragraph{Measurable functions (random variables)}
\begin{lemma}
  A mapping $f: E\to F$ and (inverse mapping) $f^{-1} (A) := \left\{ x \in E :
    f(x) \in A
  \right\}.$ Then, $f^{-1} \emptyset = \emptyset$. $f^{-1}(F) = E$. $f^{-1}
  (B\setminus C) = f^{-1} (B) \setminus f^{-1} (C)$. 
  $f^{-1}\left( \bigcup _ i B_i \right) = \bigcup _i f^{-1}(B_i)$ and $f^{-1}\left( \bigcap B_i \right) =
  \bigcap _ i f^{-1}\left( B_i \right)$.


  (set operation passes through the inverse function operation.)
  \label{lemma:measurable-function}
\end{lemma}


\begin{definition}
  $(E, \mathcal E)$, $(F, \mathcal F)$. $f: E\to F$ is "measurable" relative to
  $\mathcal E \& \mathcal F$ if $f^{-1}(B) \in \mathcal E$, $\forall B \in
  \mathcal F$. 
\end{definition}




\subsubsection{Thu: 2014-09-04 }
\paragraph{measurable functions}
(To "measure" a measurable function: just to integrate it).

\begin{proposition}
  A function $f: E \to F$ is meausrbale if and only if for some collection
  $\mathcal F_0$ with $\mathcal F = \sigma ( \mathcal F _0)$, $f^{-1}(B) \in
  \mathcal E$.
  \begin{proof}
    Necessicty is trivial; (by definitoin)

    First collect all the sets s.t. $\mathcal F_1 = \left\{ B \in \mathcal F :
      f^{-1}(B) \in \mathcal E \right\} \supset \mathcal F$. We show this by
      showing that this is a sigma algebra.  [through checking the properties of
      inverse fucntions.]
  \end{proof}
\end{proposition}

\begin{lemma}[Composition of measurable functions are measurbal]
  [2.5 Proposition]
\end{lemma}

Let $m(\mathcal E)$ note the collection of measurable functions. Abuse of
notation: let $\mathcal E$ also note $m(\mathcal E)$ since the context would be
clear.


\begin{theorem}
  
  \label{thm:something}
  \begin{proof}
    \textbf{Step 1} the sup would exist: 

    \dots

  \end{proof}
\end{theorem}

\paragraph{Approximation of measurbale functions}
\begin{lemma}
  For $r\in \R_+$,
  $d_n(r) = \sum _{k=1}^{n^{2^n}} \frac{k-1}{2^n} 1_{[\frac{k-1}{2^n},
  \frac{k}{2^n}]}(r) + n 1_{[n, \infty)}(r)$. THen $d_n(r) \to r$ as
    $n\to\infty$.
  \label{lemma:}
\end{lemma}

\begin{theorem}
  A positive function is measurable if and only if it is a limit of positive
  simple functions ($\sum_{i=1}^n a_i 1_{A_i} $ for $ a_i \in \R$ and $A_i \in
  \mathcal E$).
  \begin{proof}
    Sufficiency is given by the previous theorem. 

    Necessity: Let $f_n= d_n \circ f$ where $f_n \uparrow f$. (By construction
    of $d_n(r)$, $f_n$ is simple measurable function.)
    
  \end{proof}<
\end{theorem}

\begin{lemma}
  If $h_1, h_2, h_3 \in \mathcal E$, $h_1 + h_2, h_1 \dot h_2, \lambda h \in
  \mathcal E$ for $\lambda \in \R$.
  \label{Homework_01}
\end{lemma}

\paragraph{Decomposition of positive part and negative part of function $f$}
Let $f = f^+ - f^-$ where $f^+ = f \dots$ 

\paragraph{Monotone Class Theorem for Functions}
\begin{definition}
  For $\mathcal M$, a collection of functions is a monotone class if 
  \begin{enumerate}
    \item $1 \in \mathcal M$ (1 is the function assigning all elements in $E$ to
      $1$)
    \item $f, g \in \mathcal M_b$ $\implies$ $a f + b g \in \mathcal M_b$. 
      where $\mathcal M_b$ denote bounded functions in the set of functions
      denoted by $\mathcal M$. 

    \item $\left( f_n \right) \subset \mathcal M_ + $, and $f_n \uparrow f$
      $\implies f \in \mathcal M _ +$. 
      where $\mathcal M _ + $ denote non-negative functions in $\mathcal M$. 
  \end{enumerate}
\end{definition}

\begin{theorem}[Monotone Convergence Theorem]
  Let $\mathcal M$ be a monotone class. Suppose that for some $p$-system
  $\mathcal C$, $\mathcal E = \sigma (\mathcal C)$. 
  \[ 1 _ A \in \mathcal M, \qquad \forall A \in \mathcal C \implies \mathcal M
  \text{ includes all positive measurable functions ($\mathcal E_+$) and all bounded measurable
  functions ($\mathcal E_b$)}\]
  $1_A$ here is an indicator function


  \begin{proof}
    \textbf{Step 1:} we want to show that, for all $1 _ A \in \mathcal M$,
    $\forall A \in \mathcal E$.

    Define 

    use the defection of a monotone class to show that $\mathcal D$ is a
    $d$-system. 
    \label{homework_02}

    $\mathcal D$ being a $d$-system implies that $\mathcal \dots$

    \textbf{Step 2} Simple functions are also in $\mathcal M$. [find a reason to
    this.]

    \textbf{Step 3} By the previous theorem on the simple function, we see that
    for arbitrary $f \in \mathcal E _ + $, $\exists \left( f_n \right) \uparrow
    f$ where $f_n$ is simple measurable functions.  

    Then, by (3) in definition of monotone class (of functions) , $f \in
    \mathcal M_+$. 

    \textbf{Step 4} For $f \in \mathcal E_b$, as $f = f^+ - f^-$ by (2) in
    definition of monotone class, we have $f \in \mathcal M_b$. 
  \end{proof}
 
\end{theorem}

\begin{definition}
  $X : \Omega \to ( E , \mathcal E)$, $\sigma (X) = X^{-1} \mathcal E := \left\{
    X^{-1}(A) : A \in \mathcal E \right\}$ is called the $\sigma$-algebra
    generated by $X$. Note that $X$ here is a

    Hereby we define a new $\sigma$-algebra on $\Omega$. 
\end{definition}

\begin{proposition}
  Let $X : \Omega \to (E, \mathcal E)$ and another mapping $V : \Omega \to \bar
  \R$ belongs to $\sigma (X)$ if and only if $V = f \circ X $ for some function
  $f \in \mathcal E$. 
  \begin{proof}
    Sufficiency part is trivial; 

    Necessity: let $\mathcal M $ be the collection of all $V  f\circ X$ for some
    $f \in\mathcal E$. (This is enough to show if $Y$ is bounded measurable
    w.r.t. $\sigma (X)$, $\exists$ a bounded measurable function $f$ s.t. $Y =
    f(X)$)

    \textbf{Step 1} Show that $\mathcal M $ is a monotone class: 
    \label{homework}

    \textbf{Step 2} $\mathcal M$ includes every indicator function in $\sigma
    (X)$. The set $H \in \sigma (X)$, $H = X^{-1}(A)$ for $A \in \mathcal E$.
    Then 
    \[
      1 _ H = 1 _ A \circ X
    \]

    \textbf{Step 3 } Use MCT



  \end{proof}
\end{proposition}

\paragraph{Measures}
\begin{definition}
In a measurable space $(E, \mathcal E)$, $\m E$, for $\mu : \mathcal E \to
\bar \R_+$, if 
\begin{enumerate}[(a)]
  \item $\mu (\emptyset) = 0$
  \item $\mu ( \bigcup _ n A_n) = \sum_n \mu (A_n)$ for $A_n$'s being disjoint.
\end{enumerate}
$\mu$ is called a measure. (Note that this measure is infinite as $\mu: \mathcal E
\to \bar \R_+$, with a  bar overhead of $\R$.)
\end{definition}

\begin{proposition}
  For $A$ and $B$ being measurable sets, 
  \begin{enumerate}[(i)]
    \item (Monotonicity) $A \subset B \implies \mu(A) \le \mu (B)$;  [Implied by
      finite additivity.]
    \item (Continuity under increasing limits) $A_n \uparrow A \to \mu (A_n)
      \uparrow \mu(A)$. 

      \begin{proof}
        For $B_1 = A_1$, $B_2 = A_2 \setminus A_1$, \dots, $\bigcup _{n=1}^n B_n
        = \bigcup _{n=1}^n A_n$ [This is finite additivity.] 
        \label{Homework} [Could finish the proof by taking limits at both
        sides.]
      \end{proof}
    \item (Sub-additivity) $\mu (\bigcup _ n A_n) \le \sum (A_n)$.
  \end{enumerate}
\end{proposition}

\subparagraph{Notation}
$\mu(E) \le \infty$ $\implies $ $\mathcal M$ is a finite measure.
$\mu(E) = 1$ implies that $\mu$ is a probability measure. 

$\sigma$-finite if $\exists $ a partition\footnote{Only countable, not
necessarily finite. Each element in the partition is disjoint.} $(E_n)$ of $E$ s.t. $\sum (E_n) < \infty$. 

$\Sigma$-finite if $\exists \mu_n$ s.t. $\mu = \sum _n \mu _n $ for $\mu_n(E) <
\infty$. 

$\sigma$-finite $\implies$ $\Sigma$-finite.

\begin{theorem}
  Let $(E, \mathcal E)$ be a measurable space and measures $\mu $ and $\nu$ are
  $\mu(E) = \nu(E) < \infty$. Moreover, $\mu$ and $\nu$ agree on $\mathcal C$,
  which is a $p$-system satisfying $\mathcal E = \sigma (\m C)$, then 
  \[
    \mu = \nu
  \]
  This is why we can specify the Lebesgue measure by only assigning measure to
  the intervals. The above theorem would generalize the measure. 
  \label{thm:}
\end{theorem}<++>
Again is a consequence of monotone class theorem




\end{document} 
\substack 

\hrule
\smallskip
\hruleSince

$\ddot\smile$ %Smiling Face
$ \lfloor \rfloor$ %Tilman's special braces.


\hrule
\smallskip
\hruleSince

$\ddot\smile$ %Smiling Face
$ \lfloor \rfloor$ %Tilman's special braces.
