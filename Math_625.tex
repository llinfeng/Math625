\documentclass[11pt]{article}
\usepackage[left=.7in,right=.7in,top=.7in,bottom=.7in]{geometry}
%Set up the margins
\usepackage{amssymb,amsmath,amsthm,mathrsfs,verbatim}
\usepackage{extarrows}%% For extending the equal sign when there is   \xlongequal{<stuff>}
%\sout{} strike through!
\usepackage{marginnote}%\marginnote{}
\usepackage[colorlinks,linkcolor=blue]{hyperref}

%Create a header at the top of every page
\usepackage{fancyhdr}
\pagestyle{fancy}

\usepackage{pdfpages} %merge pdf with: \includepdfmerge{heine-borel_proof.pdf,-}
\usepackage{graphicx}%to input images \includegraphics[width=130mm]{Open_Open}
\usepackage{lipsum}
\usepackage{color}

%You can define commands for the things you use frequently.
\newcommand{\al}{\aleph}
\newcommand{\A}{\mathcal{A}}
\newcommand{\D}{\mathcal D}
\renewcommand{\c}{\mathfrak c}
\newcommand{\C}{{\mathbb C}}
\renewcommand{\L}{{\mathcal L}}
\newcommand{\M}{\mathcal M}
\newcommand{\m}{\mathcal}
\newcommand{\N}{{\mathbb N}}
\renewcommand{\P}{\mathcal {P}}
\newcommand{\Q}{{\mathbb Q}}
\newcommand{\R}{{\mathbb R}}
\newcommand{\X}{{\mathlarger{\mathcal X}}}
\newcommand{\Z}{{\mathbb Z}}
\newcommand{\To}{\Rightarrow}
\newcommand{\WLOG}{With out loss of generality}
\newcommand{\card}{\text{card }}

% The limits.
\newcommand{\limk}{\underset{k\to\infty}\lim}
\newcommand{\limn}{\underset{n\to\infty}\lim}
\newcommand{\liminfn}{\underset{n\to\infty}{\underline{\lim}}}
\newcommand{\liminfk}{\underset{k\to\infty}{\underline{\lim}}}
\newcommand{\liminfp}{\underset{p\to\infty}{\underline{\lim}}}
\newcommand{\liminfj}{\underset{j\to\infty}{\underline{\lim}}}
\newcommand{\limsupn}{\underset{n\to\infty}{\overline{\lim}}}
\newcommand{\limsupk}{\underset{k\to\infty}{\overline{\lim}}}
\newcommand{\limsupp}{\underset{p\to\infty}{\overline{\lim}}}
\newcommand{\limsupr}{\underset{r\to\infty}{\overline{\lim}}}


\newcommand{\sgn}{\text{sgn}}

% For L^p space
\newcommand{\Linfty}{{L^\infty(E)}}
\newcommand{\Lp}{{L^{p}(E)}}
\newcommand{\Lq}{{L^{q}(E)}}

% 偷懒
\renewcommand{\phi}{\varphi}
\renewcommand{\epsilon}{\varepsilon}


\usepackage{mdframed}%needed for box like theorems.\newmdtheoremenv %In the box, the footnotes are more handy! ^.^

\newtheorem{theorem}{Theorem}[subsection]
\newtheorem{acknowledgement}[theorem]{Acknowledgement}
\newtheorem{algorithm}[theorem]{Algorithm}
\newtheorem{case}[theorem]{Case}
\newtheorem{claim}[theorem]{Claim}
\newmdtheoremenv{claimbox}[theorem]{Claim}
\newtheorem{conclusion}[theorem]{Conclusion}
\newtheorem{condition}[theorem]{Condition}
\newmdtheoremenv{conjecture}[theorem]{Conjecture}
\newtheorem{corollary}[theorem]{Corollary}
\newtheorem{criterion}[theorem]{Criterion}
\newmdtheoremenv{definition}[theorem]{Definition}
\newtheorem*{defn}{Definition}
\newtheorem{example}[theorem]{Example}
\newtheorem{exercise}[theorem]{Exercise}
\newtheorem{fact}[theorem]{fact}
\newmdtheoremenv{idea}[theorem]{Idea}
\newtheorem{lemma}[theorem]{Lemma}
\newtheorem{notation}[theorem]{Notation}
\newtheorem{question}[theorem]{Question}
\newmdtheoremenv{question_sqrt}{Question}
\newtheorem{proposition}[theorem]{Proposition}
\newtheorem*{problem}{\textcolor[rgb]{1.00,0.00,0.00}{Problem}}
\newtheorem{remark}[theorem]{Remark}
\newmdtheoremenv{remark_box}[theorem]{Remark}
\newtheorem{solution}[theorem]{Solution}
\newtheorem{summary}[theorem]{Summary}
\newmdtheoremenv{setup}{Set Up}
\newmdtheoremenv{joke}{Joke}
\newmdtheoremenv{typo}{Typo Correction}
\newtheorem*{rudin}{Rudin Says}
%Textbook: elementary  : the theory of calculus
\title{Math 625\\
Professor Erhan Bayraktar}
\author{Linfeng Li \\ llinfeng@umich.edu}

\usepackage{mathrsfs} % enable people to use \mathscr{A}
\usepackage{color}
\usepackage{xcolor}
\newcommand{\hilight}[1]{\colorbox{yellow}{#1}}
\usepackage{bbm}
\usepackage{relsize} % For large symbles: \mathlarger{math_expression}
\usepackage{marvosym} % 笑脸
%\usepackage{enumitem} % For using: \begin{itemize}[leftmargin=-.5in]
\usepackage{enumerate}
%\usepackage{comment}

% To make footnote numering by section.
\makeatletter
\@addtoreset{footnote}{section}
\makeatother
\usepackage{colonequals}
\begin{document}
\maketitle
No responding to emails!
\section{Week 1}
\subsection{Tue: 2014-09-02}
\subsubsection{Basics}
Exam (in class, 90 minutes): Tue, Oct 7; Tue, Nov 11; Tue, Dec 9.   || 25\%, 30
\% and 40 \%.  
5\% attendance. 


\textbf{Textbook: } Probability and Statistics (took the class from the writer,
available through library).  
Probability with \textbf{Martingale}, the latter is the emphasis.

\textbf{Exercises:} Book has exercises, but not graded homework.

\subsubsection{Beginning: measure theory}
\paragraph{Sigma-algebra}
Given a set $E$ (a universal set), $\mathcal E$, a nonempty collection of
subsets of $E$, is called a $\sigma$-algebra if closed under complements \&
countable unions. 

\begin{itemize}
  \item The most trivial sigma-algebra  $\{\emptyset, E\}$ is called the trivial
    $\sigma$-algebra. 
\end{itemize}

\begin{definition}[$\sigma$ algebra generated by $\mathcal C$]
  Given a collection $\mathcal C$ of subsets (of $E$), $\sigma(\mathcal C)$ will
  denote the smaller $\sigma$-algebra containing $\mathcal C$. 
\end{definition}

\begin{definition}
  $\sigma$-algebra generated by open sets  is called \textbf{Borel
  $\sigma$-algebra}.
\end{definition}

\paragraph{$p$-system} A collection of $\mathcal C$ which is closed under
(finite) intersection.

``p'' for product, could also use $\pi$-system. The latter in Greek. 
\paragraph{$d$-system} A collection $\mathcal D$ is called a $d$-system if 
\begin{enumerate}[(i)]
  \item $E\in \mathcal D$, 
  \item $A, B \in \mathcal D$, $A\supset B$ $\implies$ $A\setminus B \in \mathcal D$
  \item $\left( A_n \right) \subset \mathcal D$ and $A_n \uparrow A$ $\implies$
    $A\subset \mathcal D$. 
\end{enumerate}
(``d'' for Dynkin)




Note: curly characters are for collection of sets. 
\begin{proposition}
  $\mathcal E$ is a $\sigma$-algebra if and only if it is a $p$-system and a
  $d$-system. 
% A sigma algebra is both a $p$-system and a $d$ system.
  \begin{proof}
    $\Rightarrow$ is trivial.

    $\Leftarrow$: Let $\mathcal E$ be a collection which is a $p$-system and a
    $d$-system. 
    \begin{enumerate}
      \item Closed under complements (to be a sigma algebra).
        Let $A\in  \mathcal E$ $\implies$ $E\setminus A \in \mathcal E$ by
        $(ii)$ for property of $d$-system. 
      \item Closed under finite unions: $A, B \in \mathcal E$ $\implies$
        $A\bigcup B = \left( A^c \cap B^c \right)^c$ by $1$ above and property
        of $p$-system.
      \item Closed under countable unions: for $\left( A_n \right) \subset
        \mathcal E$, $\bigcup _n A_n \in \mathcal E$? We construct an
        increasing sequence of $\left( B_n \right)$: 

        Let $B_1 = A_1$, $B_2 = A_1 \bigcup A_2 \in \mathcal E$ \dots
        $\bigcup _ n A_n = \bigcup _ n B_n $. Then by $(iii)$ for property of
        $d$-system, the conclusion follows.
    \end{enumerate}
  \end{proof}

\end{proposition}

\begin{lemma}
  For $\mathcal D$, a $d$-system, fix $D\in \mathcal D$. Define $\hat { \mathcal D}
  := \left\{ A \in \mathcal D : A\cap D \in \mathcal D \right\}$. Then, $\hat
  {\mathcal D} $ is also a $d$-system.
  \label{lemma:d-system}
  \label{hw:homework-week1}
\end{lemma}

\paragraph{Monotone Class Theorem}

[Very useful tool in showing an arbitrary collection of sets is a $\sigma$-algebra]

\begin{theorem}
  If a $d$-system contains a $p$-system, then it also contains the
  $\sigma$-algebra generated by the $p$-system.
  \label{thm:monotone-class-theorem}
  \begin{proof}
    Symbolic expression: $\mathcal C \subset \mathcal D \implies \sigma
    (\mathcal C) \subset \mathcal D$.

    \textbf{Step 1: }


    Let $\mathcal C$ be a $p$-system. $\mathcal D$ is the smallest $d$-system
    that contains $\mathcal C$. 
    \footnote{(***: little result -- the intersections of
      $d$-systems is a $d$-system [to obtain the ``smallest'']. Also, the
    ``smallest'' matters.)}

    Enough to show $\mathcal D \supset \sigma (\mathcal C)$. 

    If fact, we will show $\mathcal D$ is a $\sigma$-algebra. By proposition, it
    is enough to show it is a $p$-system. 

    \textbf{Step 2: }

    Fix $B \in \mathcal C$ and let $\mathcal D_1 := \left\{ A \in \mathcal D : A
    \bigcap B \in \mathcal D\right\}$. 1) By the lemma \ref{hw:homework-week1},
    $\mathcal D_1$ is a $d$-system. 2) $\mathcal C \subset \mathcal D_1$. 

    1) and 2) $\implies $ $\mathcal D_1 = \mathcal D$.


    \textbf{Setp 3: }

    Fix $A\in \mathcal D$, let $\mathcal D_2 := \left\{ B \in \mathcal D :
    B\bigcap A \in \mathcal D \right\}$. 

    1) by the lemma, $\mathcal D_2$ is a $d$-system. 2) by \textbf{step 2},
    $\mathcal C \subset D_2$. 

    1) and 2) $\implies$ $\mathcal D_2 = \mathcal D$. 



    \textbf{Step 1-3} gives that $\mathcal D$ is a $p$-system.




    In here, $\mathcal D = \sigma ( \mathcal C)$. [But in the theorem, this is
    not a necessary conclusion.]
  \end{proof}
\end{theorem}

\paragraph{Measurable space }
$(E, \mathcal E)$ is a measurable space. [$\mathcal E$ is a $\sigma$-algebra on
$E$.]

\paragraph{Products of measure spaces} $(E, \mathcal E)$, $(F, \mathcal F)$.
Then $\left( E \times F, \mathcal E light-product \mathcal F \right)$ where
$\times$ is regular set product; and the light-product is $\sigma \left( \text{
generated by measurable rectangles} 
\right)$

\paragraph{Measurable functions (random variables)}
\begin{lemma}
  A mapping $f: E\to F$ and (inverse mapping) $f^{-1} (A) := \left\{ x \in E :
    f(x) \in A
  \right\}.$ Then, $f^{-1} \emptyset = \emptyset$. $f^{-1}(F) = E$. $f^{-1}
  (B\setminus C) = f^{-1} (B) \setminus f^{-1} (C)$. 
  $f^{-1}\left( \bigcup _ i B_i \right) = \bigcup _i f^{-1}(B_i)$ and $f^{-1}\left( \bigcap B_i \right) =
  \bigcap _ i f^{-1}\left( B_i \right)$.


  (set operation passes through the inverse function operation.)
  \label{lemma:measurable-function}
\end{lemma}


\begin{definition}
  $(E, \mathcal E)$, $(F, \mathcal F)$. $f: E\to F$ is "measurable" relative to
  $\mathcal E \& \mathcal F$ if $f^{-1}(B) \in \mathcal E$, $\forall B \in
  \mathcal F$. 
\end{definition}




\subsubsection{Thu: 2014-09-04 }
\paragraph{measurable functions}
(To "measure" a measurable function: just to integrate it).

\begin{proposition}
  A function $f: E \to F$ is meausrbale if and only if for some collection
  $\mathcal F_0$ with $\mathcal F = \sigma ( \mathcal F _0)$, $f^{-1}(B) \in
  \mathcal E$.
  \begin{proof}
    Necessicty is trivial; (by definitoin)

    First collect all the sets s.t. $\mathcal F_1 = \left\{ B \in \mathcal F :
      f^{-1}(B) \in \mathcal E \right\} \supset \mathcal F$. We show this by
      showing that this is a sigma algebra.  [through checking the properties of
      inverse fucntions.]
    \end{proof}
  \end{proposition}

  \begin{lemma}[Composition of measurable functions are measurbal]
    [2.5 Proposition]
  \end{lemma}

  Let $m(\mathcal E)$ note the collection of measurable functions. Abuse of
  notation: let $\mathcal E$ also note $m(\mathcal E)$ since the context would be
  clear.


  \begin{theorem}

    \label{thm:something}
    \begin{proof}
      \textbf{Step 1} the sup would exist: 

      \dots

    \end{proof}
  \end{theorem}

  \paragraph{Approximation of measurbale functions}
  \begin{lemma}
    For $r\in \R_+$,
    $d_n(r) = \sum _{k=1}^{n^{2^n}} \frac{k-1}{2^n} 1_{[\frac{k-1}{2^n},
    \frac{k}{2^n}]}(r) + n 1_{[n, \infty)}(r)$. THen $d_n(r) \to r$ as
      $n\to\infty$.
      \label{lemma:}
    \end{lemma}

    \begin{theorem}
      A positive function is measurable if and only if it is a limit of positive
      simple functions ($\sum_{i=1}^n a_i 1_{A_i} $ for $ a_i \in \R$ and $A_i \in
      \mathcal E$).
      \begin{proof}
        Sufficiency is given by the previous theorem. 

        Necessity: Let $f_n= d_n \circ f$ where $f_n \uparrow f$. (By construction
        of $d_n(r)$, $f_n$ is simple measurable function.)

      \end{proof}<
    \end{theorem}

    \begin{lemma}
      If $h_1, h_2, h_3 \in \mathcal E$, $h_1 + h_2, h_1 \dot h_2, \lambda h \in
      \mathcal E$ for $\lambda \in \R$.
      \label{Homework_01}
    \end{lemma}

    \paragraph{Decomposition of positive part and negative part of function $f$}
    Let $f = f^+ - f^-$ where $f^+ = f \dots$ 

    \paragraph{Monotone Class Theorem for Functions}
    \begin{definition}
      For $\mathcal M$, a collection of functions is a monotone class if 
      \begin{enumerate}
        \item $1 \in \mathcal M$ (1 is the function assigning all elements in $E$ to
          $1$)
        \item $f, g \in \mathcal M_b$ $\implies$ $a f + b g \in \mathcal M_b$. 
          where $\mathcal M_b$ denote bounded functions in the set of functions
          denoted by $\mathcal M$. 

        \item $\left( f_n \right) \subset \mathcal M_ + $, and $f_n \uparrow f$
          $\implies f \in \mathcal M _ +$. 
          where $\mathcal M _ + $ denote non-negative functions in $\mathcal M$. 
      \end{enumerate}
    \end{definition}

    \begin{theorem}[Monotone Convergence Theorem]
      Let $\mathcal M$ be a monotone class. Suppose that for some $p$-system
      $\mathcal C$, $\mathcal E = \sigma (\mathcal C)$. 
      \[ 1 _ A \in \mathcal M, \qquad \forall A \in \mathcal C \implies \mathcal M
        \text{ includes all positive measurable functions ($\mathcal E_+$) and all bounded measurable
        functions ($\mathcal E_b$)}\]
        $1_A$ here is an indicator function


        \begin{proof}
          \textbf{Step 1:} we want to show that, for all $1 _ A \in \mathcal M$,
          $\forall A \in \mathcal E$.

          Define 

          use the defection of a monotone class to show that $\mathcal D$ is a
          $d$-system. 
          \label{homework_02}

          $\mathcal D$ being a $d$-system implies that $\mathcal \dots$

          \textbf{Step 2} Simple functions are also in $\mathcal M$. [find a reason to
          this.]

          \textbf{Step 3} By the previous theorem on the simple function, we see that
          for arbitrary $f \in \mathcal E _ + $, $\exists \left( f_n \right) \uparrow
          f$ where $f_n$ is simple measurable functions.  

          Then, by (3) in definition of monotone class (of functions) , $f \in
          \mathcal M_+$. 

          \textbf{Step 4} For $f \in \mathcal E_b$, as $f = f^+ - f^-$ by (2) in
          definition of monotone class, we have $f \in \mathcal M_b$. 
        \end{proof}

      \end{theorem}

      \begin{definition}
        $X : \Omega \to ( E , \mathcal E)$, $\sigma (X) = X^{-1} \mathcal E := \left\{
          X^{-1}(A) : A \in \mathcal E \right\}$ is called the $\sigma$-algebra
          generated by $X$. Note that $X$ here is a

          Hereby we define a new $\sigma$-algebra on $\Omega$. 
        \end{definition}

        \begin{proposition}
          Let $X : \Omega \to (E, \mathcal E)$ and another mapping $V : \Omega \to \bar
          \R$ belongs to $\sigma (X)$ if and only if $V = f \circ X $ for some function
          $f \in \mathcal E$. 
          \begin{proof}
            Sufficiency part is trivial; 

            Necessity: let $\mathcal M $ be the collection of all $V  f\circ X$ for some
            $f \in\mathcal E$. (This is enough to show if $Y$ is bounded measurable
            w.r.t. $\sigma (X)$, $\exists$ a bounded measurable function $f$ s.t. $Y =
            f(X)$)

            \textbf{Step 1} Show that $\mathcal M $ is a monotone class: 
            \label{homework}

            \textbf{Step 2} $\mathcal M$ includes every indicator function in $\sigma
            (X)$. The set $H \in \sigma (X)$, $H = X^{-1}(A)$ for $A \in \mathcal E$.
            Then 
            \[
              1 _ H = 1 _ A \circ X
            \]

            \textbf{Step 3 } Use MCT



          \end{proof}
        \end{proposition}

        \paragraph{Measures}
        \begin{definition}
          In a measurable space $(E, \mathcal E)$, $\m E$, for $\mu : \mathcal E \to
          \bar \R_+$, if 
          \begin{enumerate}[(a)]
            \item $\mu (\emptyset) = 0$
            \item $\mu ( \bigcup _ n A_n) = \sum_n \mu (A_n)$ for $A_n$'s being disjoint.
          \end{enumerate}
          $\mu$ is called a measure. (Note that this measure is infinite as $\mu: \mathcal E
          \to \bar \R_+$, with a  bar overhead of $\R$.)
        \end{definition}

        \begin{proposition}
          For $A$ and $B$ being measurable sets, 
          \begin{enumerate}[(i)]
            \item (Monotonicity) $A \subset B \implies \mu(A) \le \mu (B)$;  [Implied by
              finite additivity.]
            \item (Continuity under increasing limits) $A_n \uparrow A \to \mu (A_n)
              \uparrow \mu(A)$. 

              \begin{proof}
                For $B_1 = A_1$, $B_2 = A_2 \setminus A_1$, \dots, $\bigcup _{n=1}^n B_n
                = \bigcup _{n=1}^n A_n$ [This is finite additivity.] 
                \label{Homework} [Could finish the proof by taking limits at both
                sides.]
              \end{proof}
            \item (Sub-additivity) $\mu (\bigcup _ n A_n) \le \sum (A_n)$.
          \end{enumerate}
        \end{proposition}

        \subparagraph{Notation}
        $\mu(E) \le \infty$ $\implies $ $\mathcal M$ is a finite measure.
        $\mu(E) = 1$ implies that $\mu$ is a probability measure. 

        $\sigma$-finite if $\exists $ a partition\footnote{Only countable, not
        necessarily finite. Each element in the partition is disjoint.} $(E_n)$ of $E$ s.t. $\sum (E_n) < \infty$. 

        $\Sigma$-finite if $\exists \mu_n$ s.t. $\mu = \sum _n \mu _n $ for $\mu_n(E) <
        \infty$. 

        $\sigma$-finite $\implies$ $\Sigma$-finite.

        \begin{theorem}
          Let $(E, \mathcal E)$ be a measurable space and measures $\mu $ and $\nu$ are
          $\mu(E) = \nu(E) < \infty$. Moreover, $\mu$ and $\nu$ agree on $\mathcal C$,
          which is a $p$-system satisfying $\mathcal E = \sigma (\m C)$, then 
          \[
            \mu = \nu
          \]
          This is why we can specify the Lebesgue measure by only assigning measure to
          the intervals. The above theorem would generalize the measure. 
          \label{thm:}
        \end{theorem}
        Again is a consequence of monotone class theorem




        \section{Week 2}
        The main goal is to cover mathematical finance topics. But far away from being
        practical. Be discouraged if you are taking the course for this purpose. 
        \subsection{Tue: quick review of measure theory}
        \paragraph{(Suggested) Homework: (not to be collected)}
        Suggested homework: 1.9, 1.10, 1.11, 1.12, 1.13, 1.17; 

        Suggested homework: 2.21, 2.29, 2.31, 2.32; 

        Suggested homework: 3.11(b), 3.12; 

        \subsubsection{Specification of measures: only about uniqueness. }

        \begin{theorem}[Specificaion of measures]

          For measure: $(E, \m E)$, let $\mu$ and $\nu$ be measures with $\mu(E) = \nu (E)
          < \infty$. If $\mu$ and $\nu$ agree on $\m C$, a $p$-sysem, $\m E = \sigma ( \m
          C)$, 
          \[
            \implies \nu = \mu \text{ on } \m E
          \]


          [Uniqueness is clear. Existence is anothe story.]
          \label{thm:specification-of-measures}
          \begin{proof}
            Let $\m C$ be the $p$-system s.t. $\mu(A) = \nu (A )$ for $A \in \m C$, let 
            \[
              \m D = \{ A \in \m E: \mu(A) = \nu (A) \}  =? \m E
            \]
            (We want to use monotone class theorem) Knowing that $\m D \supset C$ (by
            statement), we need to show that $\m D$ is a $d$-system, which by Monotone
            Class Theorme (for sets) would imply that 
            \[
              \m D \supset \sigma( \supset C) = \m E
            \]
            What want to show: 
            \begin{enumerate}
              \item $E \in \m D$,  by the statement; 
              \item $A, B \in \m D$ $A \supset B$ $\implies A \setminus B \in \m D$? 
                \[
                \mu(A\setminus B) = \mu(A) - \mu(B)\]
                \[
                \nu(A\setminus B) = \nu(A) - \nu(B)\]
                So $\mu(A\setminus B ) = \nu(A\setminus B)$. 

              \item $(A_n) \subset \m D, A_n \uparrow A$, (by the sequentional continuity of measures) 
            \end{enumerate}
          \end{proof}
        \end{theorem}
        \subsubsection{Atoms, atomic \& diffuse measures}
        For a measure space (equipped with measures) $(E, \m E, \mu)$, 

        \paragraph{Atom}
        For $x\in E$ with $\mu(\{x \}) > 0$, we call $\{x\}$ an \textbf{atom}.
        \begin{proposition}
          If $\mu$ is finite, it can have at most countable many atoms.
        \end{proposition}

        \paragraph{Completeness \& Negligible (Null) Sets}
        In measure space $(E, \m E, \mu)$, let $B \in\m E$ be  a non-empty set. If
        $\mu(B) = 0$, $B$ is negligible. Furthermore, $A \subset E$ is negligible if $A
        \subset B \in \m E$ s.t. $\mu(B) = 0$>

        A measure space if ``complete'' if every negligible set is measurable.

        Though not all measure space is complete, we can \textbf{complete} (v.) the
        space as in the following proposition: 
        \begin{proposition}
          Let $\m N$ be the collection of all negligible subsets of $E$, define $\bar {\m
          E} = \sigma (\m E \bigcup \m N)$
          \begin{enumerate}[a)]
            \item $B \in \bar { \m E }$ implise that $B  = A \bigcup N, A \in \m E, N
              \in \m N$;
            \item $\bar \mu (A \bigcup N) = \mu(A)$ defines a unique measure on $\bar {
              \m E}$. 
          \end{enumerate}
          \begin{proof}
            for a) and b)
          \end{proof}<++>
        \end{proposition}

        \subsubsection{Integration}
        \paragraph{Simple Functions}
        $f =  \sum a_i 1_{A_i \in \m E}$. So the measure of a simple function is defined
        as: 
        \[
          \mu f = \int f d\mu := \sum_{i=1}^n a_i \mu(A_i)
        \]
        This is a monotone operator. 

        \paragraph{Positive functions} For $f \in \m E_+$, there exists $f_n = d_n \circ
        f$ $\uparrow f$. Define 
        \[
          \mu f := \lim \mu f_n
        \]

        \paragraph{General Measurable function}
        For $f \in \m E$ (measurable functions mapping from $E$ to $\bar \R$ (adding
        some convention): 
        \[
          f = f^+ - f^- \; \implies \; \mu f = \mu f^+ - \mu f^-
        \]
        \paragraph{Integrable}
        We say $f \in L^1 $ (integrable) if $\mu f^+ $ \& $\mu f^- < \infty$. 

        \paragraph{Remark: properties}
        \begin{enumerate}
          \item $f \ge 0 \implies \mu f \ge 0$;
          \item $f \le g$ implies $ \mu f \le \mu g$. 
        \end{enumerate}

        Notation: $\mu (f 1_A) = \int_A f d\mu$. 

        Remark: (homework) $A, B \in \m E$, $A\bigcap B = \emptyset$, $A \bigcup B = C$,
        then
        \[
          \mu (f 1_A)+ \mu (f 1_B)  = \mu (f 1_C)
        \]


        \subsubsection{Monotone Convergence Theorem}
        For $(f_n) \subset \m E_+$ and $f_n \uparrow f$, $\mu (\lim f_n) = \lim \mu
        f_n$. [due to this: we have everything well-defined.]
        \begin{proof}
          For $f \ge f_n$, by remark 2) above, $\mu f \ge \mu f_n$. Therefore, $\mu f \ge
          \lim_n \mu f_n$. 

          Now we want to show: $\mu f \le \lim_n \mu f_n$ in three steps.
          \begin{enumerate}[Step 1]
            \item Fix $b \in \R_+$ and $B \in \m E$. $f(x) > b$ for all $x \in B$.
              [$\ge$ case will be discussed later].
              \[
                \left\{ f_n > b \right\} \uparrow \left\{ f> b \right\} \implies B_n
                := B \cap \{ f_n > b \} \uparrow B
              \]
              In the limit, by the sequential continuity of measure: $\lim_n \mu(B_n)
              = \mu (B)$. So we have: 
              \[
                f_n 1_B \ge f_n 1_{B_n} \ge b 1_{B_n}
              \]
              (As $1_{B_n}$ might be $0$, so we did not write strict inequality.)
              By the remark 2), we have: 
              \[
                \mu(f_n 1_B) \ge b \mu(B_n) \implies 
                \lim _n \mu (f_n 1_B) \ge b \mu(B)
              \]
              this is the conclusion, but only given that $f(x) > b$. 

              \subitem Further: want to show $
              \lim _n \mu (f_n 1_B) \ge b \mu(B)$
              if $f \ge b$ on $B$
              \begin{itemize}
                \item[Case a] If $b = 0$, done;
                \item[Case b] For $b > 0$, have $b_m \uparrow b$. Then
                  if $f \ge b$ on $B$, we know $f > b_n$. 

                  $\forall m$,  apply step 1, we know that $\lim _n \mu(f_n1_B) \ge
                  b_m \mu(B)$. Take the limit over $m$ in the end will yield the
                  conclusion.
              \end{itemize}

            \item We can  the simple function $g$ (simple function representation) so
              that  for arbitrary $f$ (I added this arbitrary part)
              \[ f \ge g = \sum_{i=1}^m b_i 1_{B_i}
              \]
              where $\{B_i\}$  is a finite measurable partition.  Then
              \[f(x) \ge b_i \qquad \text{ for } x \in B_i
              \]
              By Step 1, 
              $\lim _n \mu (f_n 1_{B_i}) \ge b_i \mu(_iB)$
              Therefore, we conclude that: 
              \[
                \lim_n \mu(f_n) = \lim_n \sum_{i=1}^n \mu (f_n 1_{B_i}) \ge \sum b_i
                \mu(B_i)= \mu g
              \]
            \item We can ``decompose'' $f$ in the following fashion: 
              \[
                \mu f = \lim_k \mu ( d _k \circ f) 
              \]
              where $d_k \circ f$ is simple function. Apply Step 2: by construction,
              we have: 
              \[
                f \ge d_k \circ f 
              \]
              This implies that: 
              \[
                \lim _n \mu f_n \ge \mu(d_k \circ f)
              \]
              Lastly, taking $k$ to the limit will yield: 
              \[
                \lim _n \mu f_n \ge \mu f
              \]
              This is true by definition of integral: in a way that is tagged to a
              particular sequence.

          \end{enumerate} 
        \end{proof}

        \paragraph{Proposition}  \label{Prop:}
        \begin{enumerate}[1)]
          \item If $A \in \m E$ is null, $\mu (f 1 _ A) = 0$ $\forall f \in \m E$.
          \item $f , g \in \m E$ and that $f = g$ almost everywhere (a.e.), $\mu f = \mu
            g$. (By inferring to the decomposition of simple sets, 1) implies 2).
          \item If $f \in \m E_+$, $\mu f = 0$ $\implies f = 0$ a.e. 
            \begin{proof}[Proof of 3)]
              For $f \in mE _+$ and $\mu f = 0$. Let $N  = \{ f >0 \}$ and $N_k =
              \left\{ f > \epsilon _k \to 0 \right\}$. So we have: $N_k \uparrow N$. 


              $f \ge \epsilon 1 _{N_k} \implies \mu f \ge \epsilon_k \mu(N_k) \implies
              \mu(N_k) = 0
              $
            \end{proof}
        \end{enumerate}

        \subsubsection{Fatou's lemma and Corollary}
        \paragraph{Fatou's lemma}This lemma would be applied quite frequently. 
        \label{lemma:somethingelse}

        For $(f_n) \subset \m E_+$ $\implies $ $\mu ( \lim \inf f_n) \le \lim \inf \mu
        (f_n)$ (lower semi-continuity of integration). [This is the part that we lose
        whne taking $\lim \inf$.]

        \begin{proof}
          Define $g_m = \inf_{m\le n} f_n$, it is an increasing sequence where $\lim \inf f_n = \lim_m  g_m$.
          By monotone convergence theorem, we have: 
          \[\mu ( \lim \inf f_n ) = \lim_m \mu g_m
          \]

          Obseve that $g_m \le f_n, \; \forall n \ge m$. Monotonicity of integration
          implies that: 
          \[ \mu g_m \le \mu f_n, \;\; n \ge m\]
          Then, in the limit, we have: 
          \[\mu g_m \le \inf _{n \ge m} \mu f_n\]

        \end{proof}

        \paragraph{Lemma: Reverse Fatou }
        \label{lemma:fatou-reverse-sup}
        For $f_n \le g$, $g \in L^1$. Then 
        \[
          \mu (\lim \sup f_n) \ge \lim \sup \mu f_n
        \]
        (uper semi-continuity of integration). Follows from Fatou directly. 

        \paragraph{Corollary}
        \label{coro:fatou}
        $|f_n| \le g \in L^1$, and that if $(f_n)$ has a point-wise limit, by the
        previous two lemmas, 
        \[
          \mu ( \lim f_n) = \lim _ n \mu (f_n)
        \]

        \paragraph{Pointwise holds (as above), and a.e. also holds!} This marks the
        intensity of measure to null sets.

        \subsubsection{Schefle's Lemma}
        \label{lemma:schefle-negligible-differnece}
        Let $X_n, X \in L^1_+$ and $X_n \to X$ a.s. (almost surely). Then 
        \[
          \mu(|X_n - X|) \to 0 \text{ if and only if } \mu(X_n) \to \mu(X)
        \]
        (need to be a positive sequence (of functions) in $\m E$).
        \begin{proof}
          Under the assumption that $X_n, X \in L^1_+$ implies $\mu (|X_n - X|) \to 0$.
        \end{proof}

        \subsection{Thu:  }
        \subsubsection{Scheffe's lemma}
        If $f_n, f \in L^1 (E, \m E, \mu)^+$ and $f_n \to f$ a.e., then $\mu ( |f_n -
        f|) \to 0$ $\iff$ $\mu(f_n) \to \mu(f)$. 
        \begin{proof} (Showing that RHS implies LHS). 

          \textbf{Step 1}: show that the negative part converge to $0$. 
          \[
            (f_n - f)^- = \max (f - f_n, 0) \le f 
          \]
          As $f$ is integrable ($\in L^1$), by dominated convergence theorem, we have:
          $\mu\left( f_n) - f \right)^- \to 0$. 


          \textbf{Step 2}: show that the negative part converge to $0$. 
          Convergence + step 1 implies that $\mu(\left( f_n - f \right)^+ ) \to 0$ as 
          \[
            \mu\left( (f_n + f)^+ \right) = \mu(f_n - f) + \mu (( f_n - f)^-) \to 0
          \]

          The necessary part is trivial (from LHS to RHS).

        \end{proof}
        General idea: for $\mu(|g|) = \mu(g^+) + \mu(g^-)$. 


        \subsubsection{Characterization of Integral}
        If $\mu$ is measure, then all integrals have the following properties:
        \begin{enumerate}[(a)]
          \item $\mu(0) =0$
          \item $\mu ( a f+ bg ) = a \mu (f) +b \mu (g)$.
          \item For $0\le f_n \uparrow f$, $\mu(f_n) \to \mu(f)$.
        \end{enumerate}
        Remember that 


        \begin{theorem}
          See Theorem 4.21 in the book.

          \begin{proof}
            Complement to the proof in the book. 

            It remains toa show that for any $f \in \m E_+$, 

            \begin{enumerate}
              \item For $f$ being simple functions; $f  = \sum_{i=1}^n a_i 1_{A_i}$,
                $a_i \ge 0$, $A_i \in \m E$, then: 
                \[
                  \mu(f) = \sum a_i \mu(A_i) = \sum  a_i L(1_{A_i}) = L(f)
                \]
                where the last equality follows by $(b)$.
              \item In general, for $f \in \m E_+$, let $f_n \in \m E_+$ s.t. $f_n
                \uparrow f$ and $f_n$ is simple. 

                By the above step, we have: $L(f_n) = \mu (f_n)$. For $\mu(f_n)$, by
                Monotone Convergence Theorem, we have: $\mu(f_n) \to \mu (f)$. 

                On the LHS, by (c), the condition listed above in the thoerem, we have:
                $L(f_n) \to L(f)$. 

                This completes the proof.


            \end{enumerate}
          \end{proof}
          \label{thm:uniqueness-of-measure}
        \end{theorem}

        \subsubsection{Image measures}
        For measure space $(F, \m F)$ and $(E, \m E)$, let $h: F \to E$ be a measure
        w.r.t. $\m F$ and $\m E$. 

        Define $\nu \circ h^{-1}: \m E \to \bar {\R_+}$ by $\nu \circ h^{-1}(B) = \nu
        (h^{-1} B)$, $\forall B \in \m E$, where $\nu $ is a measure on $(F, \m F)$. 
        ``$\nu \circ h^{-1}$'' is a whole thing and is called ``\textbf{image of $\nu$
        under $h$.}

        \paragraph{Theorem: Change of Variable}
        $\forall f \in \m E_+$, we have $(\nu \circ h^{-1}) f  = \nu (f\circ h)$. 

        Intuition: this is nothing but a change of variable.
        \[
          \int f(y) \mu (dy) = \int f(h(x)) \nu (dx)
        \]

        \begin{proof}
          We use theorem 1 to prove it. Define $L(f) = \nu (f\circ h)$ $\forall f \in
          \m E_+$. We can check that (a), (b), (c) in Theorem
          \ref{thm:uniqueness-of-measure} holds. Then by the theorem, $\exists ! \mu $
          s.t. $\mu (f) =  L (f)$, $\forall f \in \m E_+$. Then, $
          \forall B \in \mathcal E$, 
          \[
            \mu(B) = L(1_B) = \nu (1_B \circ h) = \nu (h^{-1}(B) ) = (\nu\circ h^{-1})
            (B)
          \]
          That is, $\mu \equiv \nu \circ h^{-1}$. 

          Hence, $\forall f \in \m E_+$, 
          \[(\nu \circ h^{-1}) f = \mu (f) = L(f) = \nu (f\circ h)\]

        \end{proof}

        \subsubsection{Randon-Nikodym Theorem}
        \paragraph{Definition: }Let $\mu, \nu$ be measures on $(E, \m E)$, $\nu$ is
        absolutely continuous w.r.t. $\mu$ (\textbf{notation}: $\nu << \mu$). If $\forall A \in \m E$, $\nu
        (A) = 0 \implies \nu (A) = 0$. 

        \paragraph{Theorem 3} Assume that $\nu$ is $\sigma$-finite and $\nu << \mu$,
        then $\exists p \in \m E_+$ s.t. $\forall f \in \m E_+$, 
        \[
          \int_E f(x) \nu (dx)  = \int _  E p(x) f(x) \mu (d x)
        \]
        Moreover, $p$ is unique up to almost everywhere equivalence. (if there is a
        $p'$ that also satisfies the above equation, $p=p'$ a.e. 

        Notation-wise, we write: 
        \[\nu(dx) = p(x) \nu (dx)
        \]
        and define $\frac{\nu (dx)}{\mu(dx)} = p(x)$ as the \textbf{Randon-N
        Derivative}.

        \subsubsection{Transition Kernel}
        Let $(E, \m E)$ and $(F, \m F)$ be measure spaces. 

        \paragraph{Definition: } $k$ is a transition kernel from $(E, \m E)$ into $(F,
        \m F)$ if \footnote{
        note that $E$ is the set and $\m F$ is the $\sigma$-algebra of $F$, the set.}
        \[
          K : E\times \m F \to \bar {\R_+}
        \]
        \begin{enumerate}[(a)]
          \item For any (fixed) $B \in \m F$, the map $X \to k(x, B)$ is $\m
            E$-measurable for $x \in E$. 
          \item For nay (fixed) $x \in E$, the map: 
            $B \to K(x, B)$ is a measure on $(F, \m F)$ [note: $B \in \m F$.]
            where $B$ defines the transition probability.
        \end{enumerate}

        \paragraph{Theorem 4: }Let $K$ be a transition kernel from $(E, \m E)$ into $(F,
        \m F)$, then: 
        \begin{enumerate}[(a)]
          \item $\forall f \in \m F_+$, 
            \[ 
              K f(x) := \int _F f(y) K(x, dy), \qquad x \in E
            \]
            defines a function $Kf$ that is in $\m E_+$. 

          \item For any measure $\mu$ on $(E, \m E)$, 
            \[
              \mu K(B) := \int _E K(x, B ) \mu (dx), \qquad B \in \m F
            \]
            defines a measure $\mu K$ on $(F, \m F)$. 
          \item $(\mu K) f = \mu(K f) = \int _ E \mu (dx \int _F f(y) K(x, dy)
            $
        \end{enumerate}
        \begin{proof}

          \begin{itemize}
            \item 
              We show $(a)$ in two steps: 

              \textbf{Step 1}: if $f = \sum b_i 1 _{B_i}$, $B_i \in \m F$. 
              \[\aligned
                k f(x) & = \int _ F \sum b_i 1 _{B_i} K(x, dy) \\
                & = \sum b_i \int _F 1_{B_i} K(x, dy) \\
                & = \sum b_i K(x, B_i) \in \m E_+
              \endaligned
            \]

            \textbf{Step 2}: In  general, for $f \in \m E_+$, take $f_n$ being simple
            functions s.t. $f_n \uparrow f$. By Monotone convergence theorem, 
            \[
              Kf(x) = \lim_{n \to \infty}  K f_n(x) \in \m E_+
            \]

          \item We show (b) and (c) together in two steps: 

            \textbf{Step 1} Define $L(f) := \mu (Kf), f \in \m F_+$, show that
            (a), (b) and (c) would hold in Theorem 1.
            \begin{enumerate}[(a)]
              \item $L(0) = \mu (K 0) = \mu (0) = 0$
              \item For $f,g \in \m F_+$ and $a,b \ge 0$, then 
                \[
                  L(af + bg ) = \mu (K(af + bg)) = \mu(aKf + b Kg) 
                \]
                by linearity of integrals, due to the way in which we had
                defined $Kf$. Further, due to linearity of $\mu$: 
                \[
                  L(af + bg) = a \mu (Kf) + b\mu (kg) = a L (f) + b  L (f) 
                \]

              \item For $f_n, f \in \m F_+$, $f_n \uparrow f$. Write: 
                \[
                  L(f) = \mu (Kf)= \mu(K(\lim_n f_n) ) = \mu(\lim _n K f_n)
                \]
                due to MCT;

                since $K$ is a transition kernel and $f_n \in \m F_+$, we have: 
                \[
                  L(f) = \lim _n \mu(Kf_n) = \lim_n L(f_n)
                \]

                Then, by Theorem 1, there is a unique measure $\nu $ on $(F, \m
                F)$ s.t. $\nu (f) = L(f)$, $\forall f \in \m F_+$. Let $f \in
                1_B$ $B \in \m F$, then 
                \[
                  \nu (B) = L(1_B)  = \nu (K 1_B) = \mu (K(B))
                \]
                i.e. $\nu = \mu K$ is a measure on $(F, \m F)$. 

                Now, $\forall f \in \m F_+$, $(\mu K) f = \nu (f) = L(f) =
                \mu(Kf)$ [by definition in the begining of the proof for this
                part.

            \end{enumerate}
        \end{itemize}

    \end{proof}

    \paragraph{Propositoin:} 
    \label{prop:Tf}
    Let $K$ be a finite kernel from $(E, \epsilon)$
    into $(F, \m F)$ (i.e. $\forall x \in E$, $k(x, F) < \infty$). The forall
    positive function $f \in \m E \otimes \m F$, 
    \[
      Tf(x):= \int_F K(x, dy) f(x,y), \qquad x \in E
    \]
    defines a fucntion $Tf$ in $\m E_+$. 

    \begin{proof}
      Idea: use Monotone Class Theorem. Define $\m M := \left\{ f \in \m E
        \otimes \m F: f \text{ is positive or bounded}, Tf \in \m E \right\}$. 
        We show $\m M$ is a monotone class. For any rectangle $A\times B \in \m
        E \otimes \m F$. 
        \[
          T 1 _{A \times B } (x)  = \int _F K(x, dy) 1 _{A\times B} (x,y)
        \]
        Upon re-writing the indicator function: 
        \[
          T 1 _{A \times B } (x)  = \int _F K(x, dy) 1 _{A} (x) 1_{B}(y)
        \]
        Since $1_A(x)$ is independent of $y$, we have: 
        \[
          T 1_{A\times B}(x) = 1_A(x) K(x, B) \in \m E_+
        \]
        By monotone class thoerem, $\m M$ includes all positive function $\in \m
        E \otimes \m F$. 

    \end{proof}<++>
\end{document} 
      Labeling convention: 
      lemma & = & lemma;
      coro  & = & corollary;
      thm   & = & theorem
      eqn   & = & equation

      \substack 

      \hrule
      \smallskip
      \hruleSince

      $\ddot\smile$ %Smiling Face
      $ \lfloor \rfloor$ %Tilman's special braces.


      \hrule
      \smallskip
      \hruleSince

      $\ddot\smile$ %Smiling Face
      $ \lfloor \rfloor$ %Tilman's special braces.
